\documentclass{article}
\def\r#1{\texttt{r#1}}
\begin{document}

\section{PA2: Encrypt and Decrypt with Loops and Bitwise Operations}

In this assignment, you will write a few more programs in ARM assembly 
language by using the branching and control-flow instructions that you 
have learned about in class. 

\section{Division/Modulo}

Please complete the file \texttt{divmod.s} with a simple ARM assembly
implementation of integer division. There will be two expected inputs: the
dividend in register \r{0} and the divisor in register \r{1}. The expected
output will also have two values: the quotient in register \r{0} and the
remainder of the two numbers in register \r{1}. A sample output would look like
the following:

\begin{verbatim}
$ pa2-runner divmod.bin divmod 7 3
7 / 3 = 2
7 % 3 = 1
\end{verbatim}

For this assignment, you do not need to worry about performance, division by
zero, or negative numbers.

You may test your solution with two steps. First, we've provided a command for
you to turn your assembly into binary (it uses {\tt as}, the built-in
assembler):

\begin{verbatim}
$ make divmod.bin
\end{verbatim}

This will create a file called {\tt divmod.bin}, or report any errors the
assembler found while reading the file {\tt divmod.s}.

Then you can run your code by using the provided runner executable:

\begin{verbatim}
$ pa2-runner build/divmod.bin divmod 7 3
\end{verbatim}

You \emph{cannot} use the instructions {\tt sdiv} or {\tt udiv} in this part,
or any other part, of the assignment.

\section{A Simple Cipher}

Encryption is about making messages indecipherable without the use of a key.
In this assignment we'll implement a simple encryption scheme for text files.
You'll implement both the encryption side, which takes a key and a text file
and produces a file full of gibberish, and the decryption side, which takes a
file generated by the encryption and uses the same key to produce the original
text again.

\textbf{NOTE}: This is a bad encryption algorithm from a security point of
view. You shouldn't actually use it to keep things secret. In general, you
shouldn't write your own cryptography and put it in a product; there are
well-known algorithms and libraries for encryption. But, for exploring some
uses of bit-twiddling operations, this simple cipher is quite fun!

In this part of the exercise, you will implement a simple cipher that
encrypts and decrypts ASCII messages using a 16-bit key. Your cipher
will operate on one character at a time; after encrypting/decrypting a
single character, it will update the key used for the rest of the
message.

ASCII is a representation of characters as numbers. You can find plenty of
ASCII tables on the web with a quick search that show the value for each
letter. ASCII is quite Anglo-centric, since it focuses on the Latin alphabet,
and has stayed around for a while because it was an encoding used in early
systems. The input to your encryption algorithm will come as single-byte ASCII
characters, and you can write text files containing ASCII for encryption. We
use ASCII to keep the input space small for this sample program while still
working for interesting inputs; a more robust example would scale to an
encoding like UTF-8.

\subsection{Encryption}

In \texttt{encrypt.s}, you'll be writing the code to encrypt the character.
There will be two inputs for this algorithm: the character that will be 
encrypted and the key used for encryption, which you can assume to be 
in registers \r{0} and \r{1} respectively. After your program runs, the 
encrypted character should be in \r{0} and the updated key should be in \r{1}.

The key is 16 bits (two bytes) long, and is stored in bits 0-15 of \r{1}.

The character is 8 bits (one byte) long, and is stored in bits 0-7 of \r{0}.

First, the character needs to be encrypted, following the steps below:

\begin{enumerate} \item Save the original input character in a register for 
later use, as you'll need it when generating the new key.
\item Inspect the most significant bit of the key -- the one that corresponds
to the 1 in {\tt 0x00008000}.
\item If that bit is 1, use the upper byte of the key for the next step, if it
is 0, use the lower byte.
\item AND the key with {\tt 0x1F}.
\item EOR the input character with the key that results from this AND. This is
the encrypted character to return in \r{0}. We'll also refer to this encrypted
character in the key update below.
\end{enumerate}

For example, let's use {\tt 0x42} as the input character, and 
{\tt 0x4849} as the input key. In binary, the key would be {\tt 0100 1000 
0100 1001}. The leftmost bit is a 0, so you would use the right byte of the 
key. First, you'll need to AND the right half with {\tt 0x1F}:
\begin{center}
  \begin{tabular}{lll}
    original: & 0 1 0 0 & 1 0 0 1 \\
    0x1F: & 0 0 0 1 & 1 1 1 1 \\
    \hline \\
    AND result: & 0 0 0 0 & 1 0 0 1 \\
  \end{tabular}
\end{center}

Although the original is still unchanged, the half key we are using now holds
the value {\tt 0x09}. Notice how, by using {\tt 0x1F}, the left half byte has 
changed to have a value of {\tt 0x0}, and the right half byte is unchanged. 
Next, EOR the new key with the input character, {\tt 0x42}.
\begin{center}
  \begin{tabular}{lll}
    key: & 0 0 0 0 & 1 0 0 1 \\
    character: & 0 1 0 0 & 0 0 1 0 \\
    \hline \\
    EOR result: & 0 1 0 0 & 1 0 1 1 \\
  \end{tabular}
\end{center}

You would end up with {\tt 0x4B} as your newly encrypted character.\newline


After encrypting the character, the key needs to be updated using the 
following steps:

\begin{enumerate} \item The new key will use an integer $n$ such that the 
encrypted character modulo $n$ is equal to 5. In other words, you'll need 
to find a value for $n$ that will get a remainder of 5 when dividing the
character.

Note that, if the newly encrypted character has a value less than 11, there
is no $n$ value that satisfies the condition; in this case, use the encrypted
character as the $n$ value.
\item Multiply the old key by 4.
\item Add $n$.
\item Divide by 2.
\item Subtract the original input character, which you saved at the start.
\item Shift the register logically right until there are only two bytes left
(the upper half word is zero).
\item Reverse the value of the most significant bit of the key (that is, bit
15, the highest bit in the 2-byte key). By reverse we mean if it was 1, it becomes 
0, and vice versa.
\end{enumerate}

Continuing from the previous example, we have an encrypted character 
{\tt 0x4B} which originally held the value of {\tt 0x42}, and a key 
{\tt 0x4849}. So, in order to update the key, it would go through this 
process:
\begin{verbatim}
0x4849 * 4 = 74020
74020 + 10 = 74030
74030 / 2 = 37015
37015 - 0x42 (66) = 36949
36949 converted to hexadecimal is 0x9055
0x9055 in binary is 1001 0000 0101 0101
As it only takes up two bytes, no shift is required
Flipping the MSB gives 0x1055
\end{verbatim}

After this completes, the $n$ value would be {\tt 0xA}, and the new key 
would be {\tt 0x1055}.


At this point, you'll have the encrypted character in \r{0}, and the updated
key in \r{1}.

To test your encryption, build with:

\begin{verbatim}
$ make encrypt.bin
\end{verbatim}

\begin{verbatim}
$ pa2-runner encrypt.bin encrypt <key> <input-file> <output-file>
\end{verbatim}

Where {\tt <key>} is any two ASCII characters.

You should not write {\tt <input-file>} and {\tt <output-file>} exactly, but
instead use real filenames. We recommend making some simple text files
containing a few characters, and trying on those. For example, you might open a
file named {\tt hello.txt}, save the contents {\tt Hello!} into it, and then
run:

\begin{verbatim}
$ pa2-runner encrypt.bin encrypt ke hello.txt hello-encrypted.txt
\end{verbatim}

After running the command, the runner will place the encrypted text in the
specified file. If you open the output file in an editor, it will most likely
look like gibberish! These are the characters that were produced in \r{0} by
your algorithm, in order. You can inspect them with {\tt xxd} if you want to
look at the hex values that are stored in the file. This can be useful if you
want to write a very small two-or-three character test that you write out by
hand first, and know the expected hex values for.

\subsection{Decryption}

You should also write a decryption algorithm that inverts the process in the
file \texttt{decrypt.s}. Be sure to also reverse the check against $\#0x7F$ 
from the encryption algorithm in order to return the character to its original
state. The following commands will build and test decryption:

\begin{verbatim}
$ make encrypt.bin
\end{verbatim}

\begin{verbatim}
$ pa2-runner encrypt.bin encrypt <key> <input-file> <output-file>
\end{verbatim}

If your encryption algorithm works as expected, you can test the decryption by
using the encrypted file as input to see if you get the original text back.

\section{README}

In addition to your code, please include a README file which contains your
answers to the following questions:

The encryption and decryption algorithms are similar, though parts are
reversed. Identify one common piece of code across your implementations, and
one piece that's different. Describe in a sentence or two each why the common
piece can be the same and why the different pieces cannot be the same.

Keep your answer to under 200 words total.

\section{Commenting and Style Guide}

Every section of code, loop, or branch command needs to be commented, although
you are free to include other comments. Lines of code should not exceed 80
characters.

On including your name in files: To detect instances of academic integrity
violations in programming assignments we may use 3rd party software.  We
recommend you only include your class lab account ID (not your name or PID) in
your submissions.  Including your name and/or PID will disclose that
information to the 3rd party.

\section{Handin}

Commit and push the four files to the Github repository that was created
for you by 11:59PM on Tuesday, October 17. You can push up to one day late for
a 20\% penalty. After you push, make sure to check on Github that the files are
actually there; we will mark all of the repositories for grading a few minutes
after midnight and grade precisely what is there.

Your handin should include:

\begin{itemize}
\item Two assembly files for the cipher process, ({\tt encrypt.s}, {\tt 
decrypt.s}), and one assembly file for the divison/module function, ({\tt 
divmod.s}); 
\item A single README.txt file that has answers to the open-ended questions
above, as specified.
\end{itemize}

Note that you do not need to hand in the {\tt .bin} files.

\end{document}
